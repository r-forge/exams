
\begin{question}
Consider the following regression results:

\begin{Schunk}
\begin{Soutput}
Call:
lm(formula = y ~ x, data = d)

Residuals:
     Min       1Q   Median       3Q      Max 
-2.14867 -0.82868 -0.07472  0.66596  2.54119 

Coefficients:
             Estimate Std. Error t value Pr(>|t|)
(Intercept) 0.0001676  0.1254992   0.001    0.999
x           1.2492437  0.1241613  10.061 2.04e-14

Residual standard error: 0.9786 on 59 degrees of freedom
Multiple R-squared:  0.6318,	Adjusted R-squared:  0.6255 
F-statistic: 101.2 on 1 and 59 DF,  p-value: 2.043e-14
\end{Soutput}
\end{Schunk}

Describe how the response \texttt{y} depends on the regressor \texttt{x}.
\end{question}

\begin{solution}
The presented results describe a linear regression.

The mean of the response \texttt{y} increases with increasing \texttt{x}.

If \texttt{x} increases by $1$ unit then a change of \texttt{y} by about $1.25$ units can be expected.

Also, the effect of \texttt{x} is  significant at the $5$ percent level.
\end{solution}

%% \extype{string}
%% \exsolution{nil}
%% \exname{regression essay}
%% \exstringtype{essay|file}
%% \exextra[essay,logical]{TRUE}
%% \exextra[essay_format,character]{editor}
%% \exextra[essay_required,logical]{FALSE}
%% \exextra[essay_fieldlines,numeric]{5}
%% \exextra[essay_attachments,numeric]{1}
%% \exextra[essay_attachmentsrequired,logical]{TRUE}
%% \exmaxchars{1000, 10, 50}

