
\begin{question}
In a small city the satisfaction with the local public
transportation is evaluated. One question of interest is whether
inhabitants of the city are more satisfied with public
transportation compared to those living in the suburbs.

A survey with 250 respondents gave the following contingency table:

\begin{Schunk}
\begin{Soutput}
           Location
Evaluation  City Suburbs
  Very good   19      11
  Good        45      31
  Bad         25      66
  Very bad    11      42
\end{Soutput}
\end{Schunk}

The following table of percentages was constructed:

\begin{Schunk}
\begin{Soutput}
           Location
Evaluation  City    Suburbs
  Very good    19.0     7.3
  Good         45.0    20.7
  Bad          25.0    44.0
  Very bad     11.0    28.0
\end{Soutput}
\end{Schunk}

Which of the following statements are correct?

\begin{answerlist}
  \item The value in row~3 and column~2 in the percentage table indicates: 44 percent of the respondents in the suburbs evaluated the public transportation  as bad.
  \item The percentage table provides  row percentages.
  \item The percentage table can be easily constructed from the original contingency table: Each value is in relation to the total sample size.
  \item The value in row 4 and column~2 in the percentage table indicates: 28 percent of those, who evaluated the public transportation as very bad, live in the suburbs.
  \item The percentage table provides the satisfaction distribution for each location type.
\end{answerlist}
\end{question}

\begin{solution}

In the percentage table, the column sums are about 100 (except for possible rounding errors).
Hence, the table provides column percentages, i.e., conditional relative frequencies for satisfaction level given location type.

\begin{answerlist}
  \item True. This is the correct interpretation for column percentages.
  \item False. The  row sums are not equal to 100.
  \item False. This calculation yields total percentages. But the table provides  column percentages.
  \item False. This is an interpretation for row percentages, but the table provides  column percentages.
  \item True. The column sums are equal to 100 (except for possible rounding errors).
\end{answerlist}
\end{solution}

%% META-INFORMATION
%% \extype{mchoice}
%% \exsolution{10001}
%% \exname{Relative frequencies}
