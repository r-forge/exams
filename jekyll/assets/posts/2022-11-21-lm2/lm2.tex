
\begin{question}
\textbf{Theory:} Consider a linear regression of \texttt{y} on \texttt{x}. It is usually estimated with
which estimation technique (three-letter abbreviation)?

##ANSWER1##

This estimator yields the best linear unbiased estimator (BLUE) under the assumptions
of the Gauss-Markov theorem. Which of the following properties are required for the
errors of the linear regression model under these assumptions?

##ANSWER2##

\textbf{Application:} Using the data provided in \url{linreg.csv} estimate a
linear regression of \texttt{y} on \texttt{x}. What are the estimated parameters?

Intercept: ##ANSWER3##

Slope: ##ANSWER4##

In terms of significance at 5\% level:

##ANSWER5##

\begin{answerlist}
  \item 
  \item independent
  \item zero expectation
  \item normally distributed
  \item identically distributed
  \item homoscedastic
  \item 
  \item 
  \item \texttt{x} and \texttt{y} are not significantly correlated
  \item \texttt{y} increases significantly with \texttt{x}
  \item \texttt{y} decreases significantly with \texttt{x}
\end{answerlist}
\end{question}

\begin{solution}
\textbf{Theory:} Linear regression models are typically estimated by ordinary least squares (OLS).
The Gauss-Markov theorem establishes certain optimality properties: Namely, if the errors
have expectation zero, constant variance (homoscedastic), no autocorrelation and the
regressors are exogenous and not linearly dependent, the OLS estimator is the best linear
unbiased estimator (BLUE).

\textbf{Application:} The estimated coefficients along with their significances are reported in the
summary of the fitted regression model, showing that \texttt{y} increases significantly with \texttt{x} (at 5\% level).

\begin{Schunk}
\begin{Soutput}
Call:
lm(formula = y ~ x, data = d)

Residuals:
     Min       1Q   Median       3Q      Max 
-0.50503 -0.17149 -0.01047  0.13726  0.69840 

Coefficients:
             Estimate Std. Error t value Pr(>|t|)    
(Intercept) -0.005094   0.023993  -0.212    0.832    
x            0.558063   0.044927  12.421   <2e-16 ***
---
Signif. codes:  0 '***' 0.001 '**' 0.01 '*' 0.05 '.' 0.1 ' ' 1

Residual standard error: 0.2399 on 98 degrees of freedom
Multiple R-squared:  0.6116,	Adjusted R-squared:  0.6076 
F-statistic: 154.3 on 1 and 98 DF,  p-value: < 2.2e-16
\end{Soutput}
\end{Schunk}

\textbf{Code:} The analysis can be replicated in R using the following code.

\begin{verbatim}
## data
d <- read.csv("linreg.csv")
## regression
m <- lm(y ~ x, data = d)
summary(m)
## visualization
plot(y ~ x, data = d)
abline(m)
\end{verbatim}
\end{solution}


\exname{Linear regression}
\extype{cloze}
\exsolution{OLS|01001|-0.005|0.558|010}
\exclozetype{string|mchoice|num|num|schoice}
\extol{0.01}
