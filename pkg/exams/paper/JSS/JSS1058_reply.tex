\documentclass[a4paper]{article}
\usepackage{a4wide}
\renewcommand{\rmdefault}{phv}
\renewcommand{\sfdefault}{phv}

\begin{document}

%% encoding in exams2xyz()

%% exams_skeleton() in Section 2.3

%% evaluation policies in Sections ... and Appendix B

\textbf{\Large Reviewer 1}

\medskip

\begin{itemize}

\item {\it
In general I like the idea and article, but the article is too long and
needs to be shortened. A drawback is that the use of the software under
Ubuntu requires a lot of easily avoidable search in my linux system, search
in the internet and source code inspection:}

We appreciate the feedback and tried to address the problems experienced
in the review by bringing out the structure of the manuscript more clearly
and by providing a function \texttt{exams.skeleton()} to make getting started
easier.

The two main sections are still on usage and design of the package. We
emphasize now at the end of the introduction that persons getting started
with the \textbf{exams} package should first read Section~2 (on usage of the package)
and proceed with Section 3 (on design) only ``as needed''. Thus, we have not
shortened the manuscript because we feel that fully exploiting all features
of the package requires attention to many different details (as also indicated
by the experiences of Reviewer 1). Hence, these should be provided in the
manuscript. However, it is true that for the first steps with \textbf{exams} it
is not necessary to read all of the manuscript. To facilitate these first
steps, the function \texttt{exams.skeleton()} (or equivalently \verb|exams_skeleton()|)
creates a directory with copies of the .Rnw exercises and templates along
with a `\texttt{demo.R}' script.

Furthermore, many of the detailed comments of Reviewer 1 pertain only to
the creation of PDF exams via \texttt{exams()} and \texttt{exams2pdf()}, specifically to
customizing the {\LaTeX} master template. It is true that the manuscript does
not discuss these issues sufficiently. The reason for this is that the old
Gr\"un \& Zeileis (2009) manuscript already did this (and the new \texttt{exams2pdf()}
interface is largely backward compatible with the old \texttt{exams()} function).
Therefore, we point out more clearly at the end of Sections 2.1 and 2.2,
respectively, that PDF output is discussed in the old manuscript and not
repeated here.

\item {\it
My first problem was the errormessage ``\dots is not ASCII and does not
declare an encoding''.  The R help was useless since it was not clear how
the exams package generates its Sweave files.  Finally I changed the
encoding of my rnw files to ASCII via `\texttt{recode}'.}

We tried to look at this in more detail, using the `\verb|exams_experiments.tex|'
provided by the reviewer. However, that was not in UTF-8 encoding (anymore)
and also did not contain exercises that could be made dynamic easily. Also,
we could not compile that {\LaTeX} file without errors.

As for the encoding issue: We typically recommend to write exercises as
ASCII files (by using {\LaTeX} commands for special characters) as these can
be easily compiled on different platforms using different locale settings.
However, we tried to facilitate using different encodings. In particular,
UTF-8 now works for all \texttt{exams2}\emph{xyz}\texttt{()} interfaces. Some examples and
illustrations have been added, including support in \texttt{exams.sekelton()}.

\item {\it
Since my first `\texttt{myexam}' contained just my exam headers I got the next
errormessage `\texttt{slength >= 1L is not TRUE}'.  I guess that somehow the
missing meta information in my rnw files is responsible for this.  Meta
information seemed useless to me since my headers only contained fields
for the student name, id and general hints.  An easy possibility to insert
some static text without any exercise would have been nice. A look in the
2009 paper of Gr\"un and Zeileis revealed that somewhere in my system the
templates files used by the \textbf{exams} package are hidden.  I found them
under \texttt{/usr/local/lib/R/site-library/exams/tex}.  For writing an R package
a command `\texttt{package.skeleton}' exists, maybe something like this would be
possible also for the exams package creating some template files like
`\texttt{plain}' (solutions and exercises) and `\texttt{exam}' (only exercises). Finally, I
solved this problem with extending the question environment with optional
parameter which called \verb|\item[] ...|  and creating an empty exercise with
this optional parameter.}

First, the manuscript points out more clearly now at the end of Section~2.1
where the {\LaTeX} templates can be found and that details on their customization
can be found in Gr\"un \& Zeileis (2009).

Second, we really liked the idea of the \texttt{exams.skeleton()} function that was
now added to the package. You can simply say
\begin{verbatim}
  exams.skeleton(dir = "/path/to/my/dir")
\end{verbatim}
and this will copy all exercise .Rnw files, the corresponding templates (for
{\LaTeX}, HTML, XML) and a `\texttt{demo.R}' script to the directory indicated. The `\texttt{demo.R}'
script then shows how different types of output can be generated by using
the files provided. The \texttt{exams.skeleton()} function also allows to restrict
the `\texttt{demo.R}' script to certain types of exercises (only num or mchoice etc.)
or certain \texttt{exams2}\emph{xyz}\texttt{()} interfaces, and to use absolute paths if desired. As
mentioned above, an encoding can also be specified and then the templates
will be modified accordingly (which has been tested for UTF-8, ISO-8859-1,
and ISO-8859-15).

Third, the manuscript now recommends explicitly not to start with customization
of a specific layout in {\LaTeX} but to rather start by creating some (simple)
.Rnw exercises and rendering them into different output formats. Adaptation
of existing .Rnw exercises should hopefully be easier after creating local
copies in a directory via \texttt{exams.skeleton()}.

\item {\it
After copying the template file in my working directory I created an exam
only with `\texttt{tstat.Rnw}' with my own template file.  Integrating our logo as
a PNG graphics in my own header file leads to the errormessage ``!  LaTeX
Error: File `logo' not found.''.  After digging in the system I found under
\texttt{/tmp/RtmprrdDQN/file1c084150cdff/} the temporary directory used; identified
by time stamp :( Of course the \texttt{logo.png} was not copied to the temporary
directory.  Which meant for me I had to create my own temporary directory
where I copied the missing file.  Maybe it would helpful either to add a
parameter which allows to copy additional files or simply to copy all
files from the working directory to the temporary directory.}

Yes, that is what the argument '\texttt{inputs}' is intended for. This is pointed out
in \texttt{?exams} and \texttt{?exams2pdf} as well as Gr\"un \& Zeileis (2009).

\item {\it
After creating a subdirectory `\texttt{exercises}' with six exercises the following
call gave the errormessage that exercise files can not be found: ``Error in
\texttt{exams(myexam, template = "plain", edir = "./exercises", tdir = "./tmp")} :
The following files cannot be found: \dots''.  The exams command looked into
`\texttt{/usr/local/lib/R/site-library/exams/} \texttt{exercises/./exercises}' for the
exercise files and an absolute path for edir was necessary.}

We cannot reproduce this problem. Both absolute and relative paths work fine
for us. This is now also easily illustrated by \texttt{exams.skeleton()}. Furthermore,
consider the following example:
\begin{verbatim}
## create and change to temporary directory
td <- tempdir()
setwd(td)

## create sub-directory exercises
dir.create("exercises")

## make a copy of tstat.Rnw in exercises sub-directory
file.copy(file.path(find.package("exams"), "exercises", "tstat.Rnw"),
  file.path("exercises", "tstat-copy.Rnw"))

## the template itself is not found
exams2pdf("tstat-copy")

## but all of these work correctly
exams2pdf("tstat-copy", edir = "exercises")
exams2pdf("tstat-copy", edir = "./exercises")
exams2pdf("tstat-copy", edir = file.path(td, "exercises"))
\end{verbatim}

\item {\it
It would be
nice if the exams command would also search in subdirectories of the edir
such that one can build a directory structure containing exercises
seperated by topics.}

Thanks, this is a good idea. In case an 'edir' is specified, it is now searched
recursively for the exercise .Rnw files.

\item {\it
And again an encoding problem, but recoding did not help this time. The
problem was that the current working directory was not my test directory
but my home directory.  Therefore the plain.tex was again taken from
`\texttt{/usr/local/lib/R/site-library/exams/tex}' rather my own plain.tex from my
test directory.  After changing the working directory of R to my test
directory also the relative directory paths worked.

Adding more exercises lead again to an encoding problem. Calling recode
and cleaning my tmp directory helped now.}

Similar to some other problems reported above, we were not able to track down
these particular problems. But we hope that we have adressed it with the
improvements discussed above.

\item {\it
Replacing in my R program \texttt{exams} by \texttt{exams2pdf}/\texttt{exams2moodle} lead to
the error message ``Error: \texttt{slength >= 1L is not TRUE}''.  A internet search
showed that the Metainformation \verb|\exsolution| in the rnw-files was empty.}

Yes, the error message was too opaque. The new version of exams now reports
the following error: `no \verb|\exsolution{}| specified'

\item {\it
After correcting this error `\texttt{exams2moodle}' complained about ``Error in
\texttt{if (x[n] == sep) return(x[-n])} : Argument has length 0''.}

Sorry, we were not able to replicate this problem with the information provided
and hence were not able to fix it.

\item {\it
The only thing what really worked without any problems was the most simple
example from the paper via copy and paste.  Creating my own exam was a pain. 
It took me more then a day to get a PDF version of a current exam by myself. 
A bit more information from the exams's command, e.g.  which directories are
used, more informative error messages would have helped a lot.  Since we are
transfering ourself statistics exercises from {\LaTeX} to Moodle the
`\texttt{exams2moodle}' would have been interesting to us, but I'am not willing to
spend more time on it.}

Our feeling is that converting an existing exam layout (created without
the R package) to the \textbf{exams} package can be a challenge because the structure/setup
may be rather different and fine-tuning of the layout/formatting works somewhat
differently. As already pointed out above, in our experience it is better to
start fresh from simple exercise templates, first focusing on the content and leaving the layouting to
the package before exploring refined formatting options.

So far users of the R package had very different starting points and very different
needs when switching to the R package. Due to this heterogeneity, there is
probably not one set of documentation/examples that suits all prospective users.
We hope that \texttt{exams.skeleton()} facilitates the situation for some of them.
Furthermore, for this reason, we do provide hands-on assistance in the R-Forge
forum. Thus, if you or your colleagues are still interested in setting up
Moodle infrastructure, we encourage you to contact us in the forum and we will
try to help you.

\end{itemize}

\bigskip

\textbf{\Large Reviewer 2}

\medskip

\begin{itemize}

\item {\it
The convention of using \texttt{exams2}xyz is interesting. Why was there not a
straight-forward application of object oriented programming where there is
one common class (say `\texttt{exams}') and methods to generate the appropriate
format (much in the same way that \texttt{print} displays an object)?}

When writing the new version of the \textbf{exams} package we discussed such a setup
of first reading everything into R and then writing everything out to some
desired file again. We decided against this because reading all supplementary
graphics, data files, etc. into R would make the exams object potentially
really huge. But as the goal of the \textbf{exams} package is really to write exams
in different formats to the disk (rather than creating representations of the
exams within R), we decided not to read everything into R. Hence, we store
supplementary files only on the disk and directly weave/read/transform/write
in one go. Especially, in situations where you create a large e-learning exam
(say 1,000 copies of a dozen exercises) this seemed to be a more practical
solution, albeit not quite as elegant as a formal object-oriented approach.
 
\item {\it
Not much is said about assigning points to questions, especially
multiple choice problems. I know that this is possible in Moodle. Wouldn't
this require manual tinkering with the resulting XML files? If not, it
would be helpful to show some details in the manuscript.}

In the original version we just had \verb|\expoints{}| with the number of
points for the overall exam because for OLAT we have to do the rest of
the customization within OLAT anyway. For Moodle we had one fixed
scheme how to assign points to multiple choice items. We have now
generalised this to a user-speciefied function that implements a
particular multiple choice policy. Available policies are one wrong
answer cancels all correct ones, one wrong cancels one correct,
etc. Thanks a lot for your question which led to a much more elegant
solution.

FIXME-FL or NU
Describe policies in paper (after implementation)

FIXME-FL
One idea could be to have a generic \verb|\exextra{}{}| command for passing extra
metainformation to systems like Moodle, e.g., \verb|\exextra{answernumbering}{ABCD}|.
Alternatively, we could of course extend the set of \verb|\ex*{}| commands, e.g.,
\verb|\exanswernumbering{ABCD}| but the drawback would be that these are only recognized
by some interfaces and it would be hard to anticipate all potential tags
we may need in the future.

But maybe, Fritz, you have a better suggestion based on your experiences at Boku?

F: I currently don't need more tags beyond what we have in
\verb|read_metainfo|, and certainly not answernumbering. But \verb|\exextra| could
make a lot of sense, because one could very quickly use an extra
option for a particular driver $\rightarrow$ on 2do list for future versions?
Implement without using seems silly to me.


\item {\it
While I don't have any issue with using {\LaTeX} as the base markup
language, it may leave some unwilling to use this framework. I have seen
more willingness to accept markdown as a base format from the {\LaTeX}
uninitiated. Do the authors have any insight into this from a sociology
perspective and a technical perspective (i.e.\ translation of markdown to
other formats)?}

The reason for using {\LaTeX} is that in statistics and mathematics exams we
regularly need mathematical notation, especially for showing the correct
solutions. (This is true for most exercises in the \textbf{exams} package.)
Markdown is no alternative for this.

And for the case where no mathematical notation is needed, the markup
required is really minimal, just \verb|{question}|/\verb|{solution}| and
the \verb|\ex*| metainformation. Therefore, it is our feeling that there
is no pressing need for more explicit Markdown format.

\item {\it
Can the system accommodate multi-part questions? For example, if the
same data set is used over a series of questions, should the data be
created repeatedly for each questions? Suppose I have a simple 2-way
factorial design. I may want to ask a series of questions based on the
same data regarding: the full main effects and interaction model, the main
effects only model, model diagnostics, post-hoc comparisons, etc. How
would one structure such a problem in this system.}

When the different parts are rather small, then using ``cloze'' exercises
(as in `\texttt{boxhist.Rnw}') are a convenient option. When the parts are larger
(as in the scenario you describe above), you can simply reuse the data
that was created which is directly available in the global environment.
Consider the following simple example:
\begin{verbatim}
exams2html("boxplots")
stripchart(y ~ x, data = dat, vertical = TRUE)
\end{verbatim}
Thus, if you assure that one exercise is always carried out after the
other, the (somewhat quick \& dirty) solution is to just use the same
data again.

\item {\it
I would recommend rewriting the sentence ``Here, we focus on an
illustration how to generate different output formats form such
exercises.'' Perhaps to much alliteration for me, but it took me a minute
to parse that text.}

Thanks, rephrased now.

\end{itemize}

\end{document}
