% LaTeX-Datei fuer die Masterbelege
% Allgemeine Definitionen: 
\batchmode
\documentclass[10pt,a4paper]{article} 
\usepackage{german,graphics} 
\usepackage{amsmath}
\usepackage{amssymb,latexsym}
\setlength{\parindent}{0pt} 
\setlength{\textwidth}{21cm} 
\setlength{\textheight}{29.6cm} 
\setlength{\oddsidemargin}{-2.54cm} 
\setlength{\evensidemargin}{-2.54cm} 
\setlength{\topmargin}{-2.54cm} 
\setlength{\headheight}{0cm} 
\setlength{\headsep}{0cm} 
\setlength{\footskip}{0cm} 
\setlength{\unitlength}{1mm} 
\usepackage{chngpage}
\markboth{\textbf{\large Klausur Mathematik ID: <HEAD>}}{\large\textbf{Klausur Mathematik ID: <HEAD>}}
\pagestyle{myheadings}
\begin{document} 
\thispagestyle{empty} 
\begin{picture}(210,290) 
\thicklines 

% Positionsmarkierungen fuer Belegleser 
\put(17.5,13){\line(1,0){5}} \put(20,10.5){\line(0,1){5}} 
\put(187.5,13){\line(1,0){5}} \put(190,10.5){\line(0,1){5}} 
\put(157.5,270){\line(1,0){5}} \put(160,267.5){\line(0,1){5}} 
\put(27.5,270){\line(1,0){5}} \put(30,267.5){\line(0,1){5}} 
\put(110,157){\framebox(4,4){}} 
% die Diagonalen werden mehrmals gezeichnet damit sie dicker sind. 
\put(110,157){\line(1,1){4}} \put(110,161){\line(1,-1){4}} 
\put(110.2,157){\line(1,1){3.8}} \put(110.2,161){\line(1,-1){3.8}} 
\put(110,157.2){\line(1,1){3.8}} \put(110,160.8){\line(1,-1){3.8}} 
% Namensfeld fuer den Studenten: 
\put(72.5,244){\makebox(0,0){\textsf{pers\"onliche Daten}}} 
\put(20,198){\framebox(105,43){}} \thinlines 
\multiput(20,217)(0,12){2}{\line(1,0){90}} \thicklines 
\put(21,236){\makebox(0,5)[l]{\textsf{Zuname:}}} 
\put(21,224){\makebox(0,5)[l]{\textsf{Vorname:}}} 
\put(21,212){\makebox(0,5)[l]{\textsf{Unterschrift:}}} 
\put(123,200){\makebox(0,0)[rb]{\footnotesize{\textsf{gepr\"uft}}}} 
% Block fuer die Matrikelnummer: 
\put(159,244){\makebox(0,0){\textsf{Matrikelnummer}}} 
\put(131,233){\framebox(56,8){}} \thinlines 
\multiput(139,233)(8,0){6}{\line(0,1){1.5}} \thicklines 
\multiput(133,163)(8,0){7}{\begin{picture}(0,0) 
\multiput(0,0)(0,7){10}{\framebox(4,4){}}\end{picture}} \newcounter{nr3} 
\multiput(129,228)(0,-7){10}{\begin{picture}(0,0) 
\multiput(0,0)(60,0){2}{\makebox(0,0){\textsf{\arabic{nr3}}}}
\end{picture} \stepcounter{nr3}} 
% Allgemeine Texte und WU-Logo: 
%\put(143,266){\includegraphics{Fallbeil}} 
%\put(175,251){\includegraphics{WU-Logo}} 
\put(40,270){\makebox(0,0)[bl]{\textsf{\textbf{\LARGE{Wirtschaftsuniversit\"at Wien}}}}} \put(20,158){\makebox(0,0)[l]{\textsf{Bitte markieren 
Sie sorgsam durch Ankreuzen. Beispiel:}}} 
\put(20,147){\parbox{170mm}{\textsf{Dieser Beleg wird maschinell 
gelesen. Bitte nicht falten, nicht knicken und nicht beschmutzen. 
Verwenden Sie zum Markieren einen \textbf{blauen oder schwarzen 
Kugelschreiber}. \\ \textbf{Nur deutlich erkennbare und positionsgenaue 
Markierungen werden ausgewertet!}}}} 

% Hier muessen die entsprechenden Werte eingetragen werden !!!
% Titel und Datum der Klausur fuer den Belegkopf: 
\put(40,262){\parbox[t]{120mm}{\large{\textsf{\textbf{Klausur Mathematik <TERMIN>}}}}} 

% Kaestchen fuer Fragen (inkl. Beschriftung und Trennlinien): 
\put(21,128){\makebox(0,0){\textsf{1}}} 
\multiput(25,126)(8,0){5}{\framebox(4,4){}} 
\put(21,121){\makebox(0,0){\textsf{2}}} 
\multiput(25,119)(8,0){5}{\framebox(4,4){}} 
\put(21,114){\makebox(0,0){\textsf{3}}} 
\multiput(25,112)(8,0){5}{\framebox(4,4){}} 
\put(21,107){\makebox(0,0){\textsf{4}}} 
\multiput(25,105)(8,0){5}{\framebox(4,4){}} 
\put(21,100){\makebox(0,0){\textsf{5}}} 
\multiput(25,98)(8,0){5}{\framebox(4,4){}} 
\put(21,90){\makebox(0,0){\textsf{6}}} 
\multiput(25,88)(8,0){5}{\framebox(4,4){}} 
\put(21,83){\makebox(0,0){\textsf{7}}} 
\multiput(25,81)(8,0){5}{\framebox(4,4){}} 
\put(21,76){\makebox(0,0){\textsf{8}}} 
\multiput(25,74)(8,0){5}{\framebox(4,4){}} 
\put(21,69){\makebox(0,0){\textsf{9}}} 
\multiput(25,67)(8,0){5}{\framebox(4,4){}} 
\put(21,62){\makebox(0,0){\textsf{10}}} 
\multiput(25,60)(8,0){5}{\framebox(4,4){}} 
\put(21,52){\makebox(0,0){\textsf{11}}} 
\multiput(25,50)(8,0){5}{\framebox(4,4){}} 
\put(21,45){\makebox(0,0){\textsf{12}}} 
\multiput(25,43)(8,0){5}{\framebox(4,4){}} 
\put(21,38){\makebox(0,0){\textsf{13}}} 
\multiput(25,36)(8,0){5}{\framebox(4,4){}} 
\put(21,31){\makebox(0,0){\textsf{14}}} 
\multiput(25,29)(8,0){5}{\framebox(4,4){}} 
\put(21,24){\makebox(0,0){\textsf{15}}} 
\multiput(25,22)(8,0){5}{\framebox(4,4){}} 
\put(85,128){\makebox(0,0){\textsf{16}}} 
\multiput(89,126)(8,0){5}{\framebox(4,4){}} 
\put(85,121){\makebox(0,0){\textsf{17}}} 
\multiput(89,119)(8,0){5}{\framebox(4,4){}} 
\put(85,114){\makebox(0,0){\textsf{18}}} 
\multiput(89,112)(8,0){5}{\framebox(4,4){}} 
\put(85,107){\makebox(0,0){\textsf{19}}} 
\multiput(89,105)(8,0){5}{\framebox(4,4){}} 
\put(85,100){\makebox(0,0){\textsf{20}}} 
\multiput(89,98)(8,0){5}{\framebox(4,4){}} 
\put(27,132){\makebox(0,0)[b]{\textsf{a}}} 
\put(35,132){\makebox(0,0)[b]{\textsf{b}}} 
\put(43,132){\makebox(0,0)[b]{\textsf{c}}} 
\put(51,132){\makebox(0,0)[b]{\textsf{d}}} 
\put(59,132){\makebox(0,0)[b]{\textsf{e}}} 
\put(43,138){\makebox(0,0){\textsf{Antworten 1 -- 15}}} 
\put(72,18){\line(0,1){121}} 
\put(91,132){\makebox(0,0)[b]{\textsf{a}}} 
\put(99,132){\makebox(0,0)[b]{\textsf{b}}} 
\put(107,132){\makebox(0,0)[b]{\textsf{c}}} 
\put(115,132){\makebox(0,0)[b]{\textsf{d}}} 
\put(123,132){\makebox(0,0)[b]{\textsf{e}}} 
\put(27,18){\makebox(0,0)[b]{\textsf{a}}} 
\put(35,18){\makebox(0,0)[b]{\textsf{b}}} 
\put(43,18){\makebox(0,0)[b]{\textsf{c}}} 
\put(51,18){\makebox(0,0)[b]{\textsf{d}}} 
\put(59,18){\makebox(0,0)[b]{\textsf{e}}} 
\put(107,138){\makebox(0,0){\textsf{Antworten 16 -- 20}}} 
\put(91,94){\makebox(0,0)[b]{\textsf{a}}} 
\put(99,94){\makebox(0,0)[b]{\textsf{b}}} 
\put(107,94){\makebox(0,0)[b]{\textsf{c}}} 
\put(115,94){\makebox(0,0)[b]{\textsf{d}}} 
\put(123,94){\makebox(0,0)[b]{\textsf{e}}} 

% Block fuer die vorkodierten Werte: 
\linethickness{0.5mm} \put(20,164){\framebox(105,28){}} \thicklines  
\put(32,177){\makebox(0,0)[t]{\textsf{Belegart}}} 
\put(25,166){\framebox(14,7){}} 
\put(67,177){\makebox(0,0)[t]{\textsf{Beleg-ID}}} 
\put(46,166){\framebox(42,7){}} \put(25,183.5){\parbox{70mm}{\textsf{In 
diesem Feld d"urfen \textbf{keine} Ver"anderungen der Daten vorgenommen 
werden!}}} \thinlines \put(113,180){\line(0,1){1.5}} \thicklines 
\put(113,191){\makebox(0,0)[t]{\textsf{\textbf{Scrambling}}}} 
\put(106,180){\framebox(14,7){}} \put(116,170){\framebox(4,4){}} 
\put(114,172){\makebox(0,0)[r]{\textsf{Ersatzbeleg:}}} 
\put(32,169.5){\makebox(0,0){\Large{\textsf{020}}}}

% Hier muessen die entsprechenden Werte eingetragen werden !!!
\put(109.5,183.5){\makebox(0,0){\Large{\textsf{<CHECKSUM1>}}}}
\put(116.5,183.5){\makebox(0,0){\Large{\textsf{<CHECKSUM2>}}}}
\put(67,169.5){\makebox(0,0){\Large{\textsf{<ID>}}}}
\end{picture} 
\newpage
\thispagestyle{empty}.
\newpage
\thispagestyle{empty}
\begin{picture}(210,290) 
\thicklines 

% Positionsmarkierungen f=FCr Belegleser 
\put(17.5,13){\line(1,0){5}} \put(20,10.5){\line(0,1){5}} 
\put(187.5,13){\line(1,0){5}} \put(190,10.5){\line(0,1){5}} 
\put(157.5,270){\line(1,0){5}} \put(160,267.5){\line(0,1){5}} 
\put(27.5,270){\line(1,0){5}} \put(30,267.5){\line(0,1){5}} 
\put(110,157){\framebox(4,4){}} 
% die Diagonalen werden mehrmals gezeichnet damit sie dicker sind. 
\put(110,157){\line(1,1){4}} \put(110,161){\line(1,-1){4}} 
\put(110.2,157){\line(1,1){3.8}} \put(110.2,161){\line(1,-1){3.8}} 
\put(110,157.2){\line(1,1){3.8}} \put(110,160.8){\line(1,-1){3.8}} 
%Kreuz in Kaestchen Ersatzbeleg
%\put(114,172)
\put(116,170){\line(1,1){4}} \put(116.1,174.15){\line(1,-1){4}} 
\put(116.2,169.9){\line(1,1){3.8}} \put(116.2,174){\line(1,-1){3.8}} 
\put(116,170.2){\line(1,1){3.8}} \put(116,173.8){\line(1,-1){3.8}} 
% Namensfeld f=FCr den Studenten: 
\put(72.5,244){\makebox(0,0){\textsf{pers\"onliche Daten}}} 
\put(20,198){\framebox(105,43){}} \thinlines 
\multiput(20,217)(0,12){2}{\line(1,0){90}} \thicklines 
\put(21,236){\makebox(0,5)[l]{\textsf{Zuname:}}} 
\put(21,224){\makebox(0,5)[l]{\textsf{Vorname:}}} 
\put(21,212){\makebox(0,5)[l]{\textsf{Unterschrift:}}} 
\put(123,200){\makebox(0,0)[rb]{\footnotesize{\textsf{gepr\"uft}}}} 
% Block fuer die Matrikelnummer: 
\put(159,244){\makebox(0,0){\textsf{Matrikelnummer}}} 
\put(131,233){\framebox(56,8){}} \thinlines 
\multiput(139,233)(8,0){6}{\line(0,1){1.5}} \thicklines 
\multiput(133,163)(8,0){7}{\begin{picture}(0,0) 
\multiput(0,0)(0,7){10}{\framebox(4,4){}}\end{picture}} \newcounter{nr3a} 
\multiput(129,228)(0,-7){10}{\begin{picture}(0,0) 
\multiput(0,0)(60,0){2}{\makebox(0,0){\textsf{\arabic{nr3a}}}}
\end{picture} \stepcounter{nr3a}} 
% Allgemeine Texte und WU-Logo: 
%\put(143,266){\includegraphics{Fallbeil}} 
%\put(175,251){\includegraphics{WU-Logo}} 
\put(40,270){\makebox(0,0)[bl]{\textsf{\textbf{\LARGE{Wirtschaftsuniversit\"at Wien}}}}} \put(20,158){\makebox(0,0)[l]{\textsf{Bitte markieren 
Sie sorgsam durch Ankreuzen. Beispiel:}}} 
\put(20,147){\parbox{170mm}{\textsf{Dieser Beleg wird maschinell 
gelesen. Bitte nicht falten, nicht knicken und nicht beschmutzen. 
Verwenden Sie zum Markieren einen \textbf{blauen oder schwarzen 
Kugelschreiber}. \\ \textbf{Nur deutlich erkennbare und positionsgenaue 
Markierungen werden ausgewertet!}}}} 

% Hier muessen die entsprechenden Werte eingetragen werden !!!
% Titel und Datum der Klausur fuer den Belegkopf: 
\put(40,262){\parbox[t]{120mm}{\large{\textsf{\textbf{Klausur Mathematik <TERMIN>}}}}} 

% Kaestchen fuer Fragen (inkl. Beschriftung und Trennlinien): 
\put(21,128){\makebox(0,0){\textsf{1}}} 
\multiput(25,126)(8,0){5}{\framebox(4,4){}} 
\put(21,121){\makebox(0,0){\textsf{2}}} 
\multiput(25,119)(8,0){5}{\framebox(4,4){}} 
\put(21,114){\makebox(0,0){\textsf{3}}} 
\multiput(25,112)(8,0){5}{\framebox(4,4){}} 
\put(21,107){\makebox(0,0){\textsf{4}}} 
\multiput(25,105)(8,0){5}{\framebox(4,4){}} 
\put(21,100){\makebox(0,0){\textsf{5}}} 
\multiput(25,98)(8,0){5}{\framebox(4,4){}} 
\put(21,90){\makebox(0,0){\textsf{6}}} 
\multiput(25,88)(8,0){5}{\framebox(4,4){}} 
\put(21,83){\makebox(0,0){\textsf{7}}} 
\multiput(25,81)(8,0){5}{\framebox(4,4){}} 
\put(21,76){\makebox(0,0){\textsf{8}}} 
\multiput(25,74)(8,0){5}{\framebox(4,4){}} 
\put(21,69){\makebox(0,0){\textsf{9}}} 
\multiput(25,67)(8,0){5}{\framebox(4,4){}} 
\put(21,62){\makebox(0,0){\textsf{10}}} 
\multiput(25,60)(8,0){5}{\framebox(4,4){}} 
\put(21,52){\makebox(0,0){\textsf{11}}} 
\multiput(25,50)(8,0){5}{\framebox(4,4){}} 
\put(21,45){\makebox(0,0){\textsf{12}}} 
\multiput(25,43)(8,0){5}{\framebox(4,4){}} 
\put(21,38){\makebox(0,0){\textsf{13}}} 
\multiput(25,36)(8,0){5}{\framebox(4,4){}} 
\put(21,31){\makebox(0,0){\textsf{14}}} 
\multiput(25,29)(8,0){5}{\framebox(4,4){}} 
\put(21,24){\makebox(0,0){\textsf{15}}} 
\multiput(25,22)(8,0){5}{\framebox(4,4){}} 
\put(85,128){\makebox(0,0){\textsf{16}}} 
\multiput(89,126)(8,0){5}{\framebox(4,4){}} 
\put(85,121){\makebox(0,0){\textsf{17}}} 
\multiput(89,119)(8,0){5}{\framebox(4,4){}} 
\put(85,114){\makebox(0,0){\textsf{18}}} 
\multiput(89,112)(8,0){5}{\framebox(4,4){}} 
\put(85,107){\makebox(0,0){\textsf{19}}} 
\multiput(89,105)(8,0){5}{\framebox(4,4){}} 
\put(85,100){\makebox(0,0){\textsf{20}}} 
\multiput(89,98)(8,0){5}{\framebox(4,4){}} 
\put(27,132){\makebox(0,0)[b]{\textsf{a}}} 
\put(35,132){\makebox(0,0)[b]{\textsf{b}}} 
\put(43,132){\makebox(0,0)[b]{\textsf{c}}} 
\put(51,132){\makebox(0,0)[b]{\textsf{d}}} 
\put(59,132){\makebox(0,0)[b]{\textsf{e}}} 
\put(43,138){\makebox(0,0){\textsf{Antworten 1 - 15}}} 
\put(72,18){\line(0,1){121}} 
\put(91,132){\makebox(0,0)[b]{\textsf{a}}} 
\put(99,132){\makebox(0,0)[b]{\textsf{b}}} 
\put(107,132){\makebox(0,0)[b]{\textsf{c}}} 
\put(115,132){\makebox(0,0)[b]{\textsf{d}}} 
\put(123,132){\makebox(0,0)[b]{\textsf{e}}} 
\put(27,18){\makebox(0,0)[b]{\textsf{a}}} 
\put(35,18){\makebox(0,0)[b]{\textsf{b}}} 
\put(43,18){\makebox(0,0)[b]{\textsf{c}}} 
\put(51,18){\makebox(0,0)[b]{\textsf{d}}} 
\put(59,18){\makebox(0,0)[b]{\textsf{e}}} 
\put(107,138){\makebox(0,0){\textsf{Antworten 16 - 20}}} 
\put(91,94){\makebox(0,0)[b]{\textsf{a}}} 
\put(99,94){\makebox(0,0)[b]{\textsf{b}}} 
\put(107,94){\makebox(0,0)[b]{\textsf{c}}} 
\put(115,94){\makebox(0,0)[b]{\textsf{d}}} 
\put(123,94){\makebox(0,0)[b]{\textsf{e}}} 

% Block fuer die vorkodierten Werte: 
\linethickness{0.5mm} \put(20,164){\framebox(105,28){}} \thicklines  
\put(32,177){\makebox(0,0)[t]{\textsf{Belegart}}} 
\put(25,166){\framebox(14,7){}} 
\put(67,177){\makebox(0,0)[t]{\textsf{Beleg-ID}}} 
\put(46,166){\framebox(42,7){}} \put(25,183.5){\parbox{70mm}{\textsf{In 
diesem Feld d"urfen \textbf{keine} Ver"anderungen der Daten vorgenommen 
werden!}}} \thinlines \put(113,180){\line(0,1){1.5}} \thicklines 
\put(113,191){\makebox(0,0)[t]{\textsf{\textbf{Scrambling}}}} 
\put(106,180){\framebox(14,7){}} \put(116,170){\framebox(4,4){}} 
\put(114,172){\makebox(0,0)[r]{\textsf{Ersatzbeleg:}}} 
\put(32,169.5){\makebox(0,0){\Large{\textsf{020}}}}

% Hier muessen die entsprechenden Werte eingetragen werden !!!
\put(109.5,183.5){\makebox(0,0){\Large{\textsf{<CHECKSUM1>}}}}
\put(116.5,183.5){\makebox(0,0){\Large{\textsf{<CHECKSUM2>}}}}
\put(67,169.5){\makebox(0,0){\Large{\textsf{<ID>}}}}
\end{picture} 
\newpage
\thispagestyle{empty}.
\newpage
\changepage{-5cm}{-3cm}{1.5cm}{1.5cm}{1cm}{1cm}{1cm}{0.5cm}{0cm}
