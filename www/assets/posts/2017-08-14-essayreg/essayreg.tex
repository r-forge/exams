
\begin{question}
Consider the following regression results:

\begin{Schunk}
\begin{Soutput}
Call:
lm(formula = log(y) ~ log(x), data = d)

Residuals:
    Min      1Q  Median      3Q     Max 
-6.6119 -1.4477  0.1735  1.5365  4.8160 

Coefficients:
            Estimate Std. Error t value Pr(>|t|)
(Intercept)   0.1264     0.2520   0.501    0.618
log(x)        0.2870     0.2279   1.259    0.212

Residual standard error: 2.251 on 79 degrees of freedom
Multiple R-squared:  0.01967,	Adjusted R-squared:  0.007263 
F-statistic: 1.585 on 1 and 79 DF,  p-value: 0.2117
\end{Soutput}
\end{Schunk}

Describe how the response \texttt{y} depends on the regressor \texttt{x}.
\end{question}

\begin{solution}
The presented results describe a log-log regression.

The mean of the response \texttt{y} increases with increasing \texttt{x}.

If \texttt{x} increases by $1$ percent then a change of \texttt{y} by about $0.29$ percent can be expected.

However, the effect of \texttt{x} is \emph{not} significant at the $5$ percent level.
\end{solution}

%% \extype{string}
%% \exsolution{nil}
%% \exname{regression essay}
%% \exextra[essay,logical]{TRUE}
%% \exextra[essay_format,character]{editor}
%% \exextra[essay_required,logical]{FALSE}
%% \exextra[essay_fieldlines,numeric]{5}
%% \exextra[essay_attachments,numeric]{1}
%% \exextra[essay_attachmentsrequired,logical]{FALSE}


