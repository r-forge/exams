
\begin{question}
  In a small city the satisfaction with the local public
  transportation is evaluated. One question of interest is whether
  inhabitants of the city centre are more satisfied with public
  transportation compared to those living in the suburbs.

  A survey with 250 respondents gave the following contingency table:
\begin{Schunk}
\begin{Soutput}
           Location
Evaluation  city centre suburbs
  very good          24      13
  good               38      31
  bad                30      61
  very bad            8      45
\end{Soutput}
\end{Schunk}

The following table of percentages was constructed:
\begin{Schunk}
\begin{Soutput}
           Location
Evaluation  city centre suburbs    
  very good        64.9        35.1
  good             55.1        44.9
  bad              33.0        67.0
  very bad         15.1        84.9
\end{Soutput}
\end{Schunk}

Which of the following statements are correct?

\begin{answerlist}
  \item The percentage table can be easily constructed from the original contingency table: percentages are calculated for each column.
  \item The percentage table contains  row percentages.
  \item The value in row~2 and column~2 in the percentage table indicates: 44.9 percentage of those, who evaluated the public transportation as good live in the suburbs.
  \item The percentage table gives the satisfaction distribution for each location type.
  \item The value in row~1 and column~2 in the percentage table indicates: 35.1 percent of those living in the suburbs evaluated the public transportation  as very good.
\end{answerlist}
\end{question}

\begin{solution}

In the percentage table, the row sums are about 100
(except for possible rounding errors). Hence, the table provides
row percentages, i.e.,
conditional proportions for location given satisfaction level.

\begin{answerlist}
  \item False. This evaluation gives column percentages. But the table gives  row percentages.
  \item True. The  row sums  are about equal to 100 (except for possible rounding errors).
  \item True. This is the correct interpretation of row percentages.
  \item False. The column sums do not give 100.
  \item False. This is an interpretation of column percentages, but the table gives  row percentages.
\end{answerlist}
\end{solution}

%% META-INFORMATION
%% \extype{mchoice}
%% \exsolution{01100}
%% \exname{Relative frequencies}
